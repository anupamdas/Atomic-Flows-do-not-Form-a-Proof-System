\documentclass[a4paper]{amsart}


\usepackage[lutzsyntax,noxy]{virginialake}
\usepackage{amsmath}

\newtheorem{theorem}{Theorem}
\newtheorem{maintheorem}[theorem]{Main Theorem}
\newtheorem{observation}[theorem]{Observation}
\newtheorem{corollary}[theorem]{Corollary}
\newtheorem{lemma}[theorem]{Lemma}
\newtheorem{proposition}[theorem]{Proposition}
\newtheorem{conjecture}[theorem]{Conjecture}

\theoremstyle{definition}
\newtheorem{definition}[theorem]{Definition}
\newtheorem{example}[theorem]{Example}
\newtheorem{notation}[theorem]{Notation}
\newtheorem{convention}[theorem]{Convention}
\newtheorem{remark}[theorem]{Remark}
\newtheorem{discussion}[theorem]{Discussion}

\newcommand{\fff}{\mathsf f}
\newcommand{\ttt}{\mathsf t}

\newcommand{\ai  }{\mathsf{ai}}
\newcommand{\aw  }{\mathsf{aw}}
\newcommand{\ac  }{\mathsf{ac}}
\newcommand{\aid }{{\ai{\downarrow}}}
\newcommand{\awd }{{\aw{\downarrow}}}
\newcommand{\acd }{{\ac{\downarrow}}}
\newcommand{\aiu }{{\ai{\uparrow}}}
\newcommand{\awu }{{\aw{\uparrow}}}
\newcommand{\acu }{{\ac{\uparrow}}}
\newcommand{\swi }{\mathsf{s}}
\newcommand{\med }{\mathsf{m}}
\newcommand{\smix }{\mathsf{sm}}
\newcommand{\mix }{\mathsf{mix}}

\newcommand  {\gw  }{\mathsf w}
\newcommand  {\gwd }{{\gw{\downarrow}}}
\newcommand  {\gwu }{{\gw{\uparrow}}}
\newcommand  {\gc  }{\mathsf c}
\newcommand  {\gcdown }{{\gc{\downarrow}}}
\newcommand  {\gcu }{{\gc{\uparrow}}}
\newcommand  {\gi  }{\mathsf i}
\newcommand  {\giu }{{\gi{\uparrow}}}
\newcommand  {\gid }{{\gi{\downarrow}}}

\newcommand{\SKS}{\mathsf{SKS}}
\newcommand{\KS}{\mathsf{KS}}
\newcommand{\SKSg}{\mathsf{SKSg}}
\newcommand{\KSg}{\mathsf{KSg}}

\begin{document}

\title{Atomic Flows do not Form a Proof System}
\author{Anupam Das}
\date{\today}

\begin{abstract}
We prove that, given certain cryptographic assumptions, atomic flows do not form a proof system when working with units.
\end{abstract}

\maketitle

\section{Introduction}

\section{Deep Inference and Atomic Flows}

\section{Admissibility of Non-Trivial Weakenings}
We show that, when an atom and its dual appear in a tautology, each proof of that tautology can be polynomially transformed into one where each occurrence of the atom and its dual are introduced by an identity step.

\begin{definition}[Trivial Weakenings]
If a tautology $\xi\{a^\epsilon\}$ has no occurrence of $\bar{a}$, and $\Phi$ is a proof of $\xi\{a^\epsilon\}$, we say that $a^\epsilon$ is a \emph{trivial weakening} in $\Phi$.
\end{definition}

\subsection{Supermix in Linear Deep Inference Systems}

\begin{definition}[Supermix]
We define the supermix rule below:
\[
\vlinf{\smix}{}{\vls[A.\bigwedge_{i}^n B_i ]}{\vls(A.\bigvee_{i}^n B_i )}
\]
For the special case when $n=1$, it coincides with the usual mix rule. Supermix is sound, and trivially derivable for $\{\gwd, \gwu\}$.
\end{definition}

\begin{proposition}\label{mix}
There is a derivation from $\fff$ to $\ttt$ for $\{\med\}$.
\end{proposition}
\begin{proof}
\[
\vlderivation{
\vlin{=}{}{\ttt}{
\vlin{\med}{}{\vls(\vlinf{=}{}{\ttt}{\vls[\fff.\ttt]}.\vlinf{=}{}{\ttt}{\vls[\fff.\ttt]})}{
\vlin{=}{}{\vls[\vlinf{=}{}{\vls(\fff.\ttt)}{\fff}.\vlinf{=}{}{\vls(\fff.\ttt)}{\fff}]}{\vlhy{\fff}}
}
}
}
\]
\end{proof}

\begin{lemma}\label{Triv}
There is a derivation from $\bigvee_{i}^n B_i$ to $\ttt\vlor\bigwedge_{i}^n B_i$ for $\{\swi,\med\}$.
\end{lemma}
\begin{proof}
We proceed by induction on $n$.

\emph{Base Case}: By Prop.\ \ref{mix} we have $\vls[\vliqf{\med}{}{\ttt}{\fff}.B]$.

\emph{Inductive Step}: Suppose there are such derivations $\Phi_r$ for $r<n$. Define:
\[
\Phi_n \quad\equiv \quad
\vlderivation{
\vliq{\swi}{}{\vls[\vlinf{=}{}{\ttt}{\vls[\ttt.\ttt.\ttt]}.(B_n .\bigwedge_i B_i )]}{
\vlin{\med}{}{\vls([\ttt.B_n ].[\ttt.\ttt.\bigwedge_i B_i ])}{\vlhy{\vls[\vlinf{=}{}{\ttt\vlan B_n }{B_n }.\vlinf{=}{}{\vls(\ttt.[\ttt.\bigwedge_i B_i ])}{\vlder{\Phi_{n-1}}{\{\swi,\med\}}{\ttt\vlor\bigwedge_i B_i }{\bigvee_i B_i }}]}}
}
}
\]
\end{proof}


\begin{theorem}
Supermix is derivable for $\{\swi,\med\}$.
\end{theorem}

\begin{proof}
Let $\Phi_n$ be the derivations constructed in Lemma \ref{Triv}.
\[
\vlderivation{
\vlin{\swi}{}{\vls[\vlinf{=}{}{A}{A\vlan\ttt}.\bigwedge_{i}^n B_i ]}{\vlhy{\vls(A. \vlder{\Phi_n }{\{\swi,\med\}}{\ttt\vlor\bigwedge_{i}^n B_i }{\bigvee_{i}^n B_i })}}
}
\]
\end{proof}

\begin{remark}
All derivations in this section have used only the logical rules, and so create no new vertices in an atomic flow. Their flows are just edges in parallel.
\end{remark}

\subsection{Converting Non-Trivial Weakenings to Identities}

\begin{proposition}
Every $\KS$-proof can be polynomially transformed into a unit-free proof with shape $(\gwd,\aid,\{\swi,\med\},\acd)$.
\end{proposition}

\begin{lemma}[Atomic Weakening]\label{AtWk}
In a unit-free system, weakening can be made atomic with only quadratic increase in proof size.
\end{lemma}

\begin{theorem}
There is a proof in $\KS\setminus\{\awd\}$ of each expression $\bigwedge_i [\ttt \vlor A_i ]$ such that no atom occurs trivially.
\end{theorem}

\begin{proof}
By Lemma \ref{AtWk} we can assume that each $A_i$ is a disjunction of atoms. We first undergo a process of `balancing' the formula; by this we mean we applying atomic contractions until there are equally many positive and negative occurrences of each atom. Suppose $\vls(A.[B.a].[C.\bar{a}])$ is a formula in this form and consider the following derivation in $\KS\setminus\{\awd\}$:
\[
\vls(A.
\vlderivation{
\vlin{\med}{}{\vls([B.a].[C.\bar{a}])}{
\vlin{\smix}{}{\vls[(B.C).(a.\bar{a})]}{
\vlin{=}{}{\vls(B.C.\vlinf{\aid}{}{\vls[a.\bar{a}]}{\ttt})}{\vlhy{B\vlan C}}
}
}
}
)
\]

Observe that the premiss is $A\vlan B\vlan C$ which is also in this form but has strictly fewer atom occurrences. Therefore we can just repeat the above procedure, chaining the derivations together to obtain a proof.
\end{proof}

\section{Identity Normal Form of a Proof}
In this section we describe how to polynomially transform a $\KS$-proof into one whose flow is in some way easy to generate, as we will see in the next section.

The following trick appears in Stra\ss burger CITE in some variation.

\begin{definition}[Balancing Trick]

\end{definition}

\section{Main Result}

\subsection{Automatisability and Feasible Interpolation}

\begin{definition}[Feasible Interpolation]

\end{definition}

\begin{definition}[Automatisability]

\end{definition}

\begin{theorem}[Impaglazzio]
If a system is automatisable then it admits feasible interpolation.
\end{theorem}

\begin{theorem}[Bonet et al.]
If integer factoring is hard for $P/poly$ then $LK$ does not admit feasible interpolation.
\end{theorem}

\begin{observation}[Jerabek]
$\KS$ admits feasible interpolation just if $LK$ does.
\end{observation}

\begin{corollary}
If integer factoring is hard for $P/poly$ then $\KS$ does not admit feasible interpolation, and hence is not automatisable.
\end{corollary}

\subsection{Automatisability Algorithm From Polynomial-Time Correctness Criterion for Atomic Flows}

\section{Conclusions}
\subsection{Flow Design}
\subsection{Unit-Free Deep Inference}
\subsection{The Linear Fragment of Propositional Logic}

\bibliographystyle{alpha}
\bibliography{biblio}

\end{document}
